\chapter{Integrointi tasossa}

\begin{variable}
    Olkoon $X$ joukko, ja $d$ $X$:n metriikka.
    Nyt $(X, d)$ on metrinen avaruus ja täten sillä on metriikan indusoima topologia.
\end{variable}

% TODO siirrä sisäpiste, joukko jutut muualle

\begin{definition}[Sisäpiste]
    \label{sisäpiste}
    Olkoon $A \subset X$.

    Piste $x$ on joukon $A$ \emph{sisäpiste}, jos

    \[
        \exists r \gt 0, B_d(x, r) \subset A
    \]

    missä $B_d(x, r)$ on metrisen avaruuden $(X, d)$ avoin $r$-säteinen $x$-keskeinen kuula.
\end{definition}

\begin{definition}[Sisäjoukko (sisus)]
    \label{sisäjoukko}
    \uses{sisäpiste}
    Olkoon $A \subset X$.

    $A$:n \emph{sisäjoukko} $\interior A$ on $A$:n sisäpisteiden joukko.
\end{definition}

\begin{definition}[Pistevieras]
    \label{pistevieras}
    Joukot $C_i \subset \R$ ovat \emph{pistevieraita} jos leikkausjoukot $\C_i \inter C_j = \emptyset$ kaikilla $i \neq j$, lukuunottamatta äärellistä määrää "kulmapisteitä".

    Huom: tämä kurssin käyttämä määritelmä on epäpätevä, ks. määritelmä \ref{reunavieras}
\end{definition}

\begin{variable}
    \label{muuttuja_C_i}
    Olkoon $C_i \subset X$ numeroituva jono $X$ osajoukkoja.
\end{variable}

\begin{definition}[Reunavieras]
    \label{reunavieras}
    \uses{sisäjoukko}
    Joukot $C_i$ (muuttujasta \ref{muuttuja_C_i}) ovat \emph{reunavieraita} jos leikkausjoukot $\interior C_i \inter \interior C_j = \emptyset$ kaikilla $i \neq j$.
\end{definition}

\begin{remark}
    Määritelmä \ref{reunavieras} on yleistys kurssin määritelmästä \ref{pistevieras}, sillä pistevierautta käytetään kurssilla vain $\R^1$ osajoukoista (joiden reuna koostuu $\R^1$ pisteistä).
\end{remark}

\begin{theorem}[Reunavieras (vaihtoehtoinen määritelmä)]\
    \label{reunavieras_vaiht}
    \uses{reuna}
    Joukot $C_i$ ovat \emph{reunavieraita} jos ja vain jos $C_i \inter C_j \subset \partial C_i \union \partial C_j$ kaikilla $i \neq j$.
\end{theorem}

\begin{definition}[Joukon peite]
    \label{peite}
    Olkoon $S$ joukko.

    Perhe $P \subset \powerset (S)$ on \emph{joukon} $S$ \emph{peite}, jos se koostuu $S$:n osajoukoista, jotka yhdessä sisältävät kaikki $S$ alkiot.
    Täsmällisemmin,
    \[
    \bigcup_{p \in P} p = s.
    \]
\end{definition}

\begin{definition}[Joukon ositus]
    \label{ositus}
    \uses{peite}
    Olkoon $S$ joukko.

    Peite $P$ on \emph{joukon} $S$ \emph{ositus}, jos se koostuu epätyhjistä erillisistä joukoista.
    Täsmällisemmin,

    \begin{itemize}
        \item $\emptyset \not\in P$
        \item $p_1 \inter p_2 = \emptyset$ kaikilla eri $p_1, p_2 \in P$.
    \end{itemize}
\end{definition}

\begin{definition}[Reunavieras ositus]
    \label{reunavieras_ositus}
    \uses{peite, reunavieras}
    Olkoon $S \subset X$.

    $S$ numeroituva peite $C_i$ on $S$ \emph{reunavieras ositus}, jos $C_i$ on reunavieras, ja lisäksi $C_i \neq \emptyset$.
\end{definition}

\begin{theorem}[Ositus on reunavieras ositus]
    \label{ositus_reunavieras}
    \uses{reunavieras_ositus, ositus}
\end{theorem}
\begin{proof}
    Osituksen joukot ovat erillisiä, joten ne ovat reunavieraita.
\end{proof}

\section{Integraalin määrittely suorakaiteen yli}

\begin{definition}[Suorakaiteen ositus]
    \label{suorakaiteen_ositus}
    \uses{reunavieras_ositus}
    Olkoon $n, m \in \N_1$ ja $D = [a, b] \times [c, d]$.

    Suorakaiteen $D$ \emph{ositus eli jako} $\set{R_{i j}}_{1 \le i \le m, 1 \le j \le n}$ muodostuu välien $[a, b]$ ja $[c, d]$ osituksista: $a = x_0 \lt \dots \lt x_m = b$, $c = y_0 \lt \dots \lt y_n = d$, missä suorakaiteet $R_{i j} = [x_{(i - 1)}, x_i] \times [y_{(j - 1)}, y_j]$ muodostavat joukon $D$ reunavieraan osituksen.
\end{definition}

\begin{remark}
    \uses{suorakaiteen_ositus}
    Huomaa, että jos $\set{R_{i j}}$ on ositus, niin $R_{i j}$ on suorakaide, ja ei ole ositus, toisin kuin kurssikirjassa on käytäntönä.

    Tästedes eroamme kurssikirjasta käyttämällä $R_{i j}$ sijaan $\set{R_{i j}}$ kun kyseessä on ositus.
\end{remark}

\begin{theorem}[Suorakaiteen ositus on reunavieras ositus]
    \label{suorakaiteen_ositus_reunavieras}
    \uses{suorakaiteen_ositus}
\end{theorem}
\begin{proof}
    Ilman yleistyksen menetystä $R_{i j} = [x_{(i - 1)}, x_i] \times [y_{(j - 1)}, y_j]$ leikkaa ainoastaan sellaisia $R_{k l}$, missä $k = i + 1$ tai $l = j + 1$.
    
    Huomataan, että $\interior R_{i j} = (x_{(i - 1)}, x_i) \times (y_{(j - 1)}, y_j)$, joten selvästi $\interior R_{i j} \inter \interior R_{k l} = \emptyset$.
\end{proof}

\begin{definition}[Suorakaiteen pinta-ala]
    \label{suorakaiteen_pinta_ala}
    Olkoon $a \leq b$ ja $c \leq d$.
    Suorakaiteen $R = [a, b] \times [c, d]$ \emph{pinta-ala} $\Delta R := (b - a)(d - c)$.
\end{definition}

\begin{definition}[Suorakaiteen läpimitta]
    \label{suorakaiteen_läpimitta}
    Olkoon $a \leq b$ ja $c \leq d$.
    Suorakaiteen $R = [a, b] \times [c, d]$ \emph{läpimitta} $d(R) = \left((b - a)^2 + (d - c)^2\right)^{1/2}$.
\end{definition}

\begin{definition}[Jaon normi]
    \label{jaon_normi}
    \uses{suorakaiteen_läpimitta, suorakaiteen_ositus}
    Olkoon $\set{R_{i j}}$ suorakaiteen $D$ ositus.

    \emph{Jaon normilla} $\norm{\set{R_{i j}}}$ tarkoitetaan suorakaiteiden $R_{i j}$ suurinta läpimittaa, eli
    
    \[
    \norm{\set{R_{i j}}} := \max_{i, j} d(R_{i j}).
    \]
\end{definition}

\begin{variable}
    Olkoon $D$ suorakaide ja $f : D \to \R$ rajoitettu.
\end{variable}

\begin{definition}[Riemannin summa]
    \label{riemannin_summa}
    \uses{suorakaiteen_ositus, suorakaiteen_pinta_ala}
    Olkoon $\set{R_{i j}}_{1 \le i \le m, 1 \le j \le n}$ suorakaiteen $D$ ositus ja $\set{\xi_{i j}}$ siten, että $\xi_{i j} \in R_{i j}$ kaikilla $i$ ja $j$.

    Funktion $f$ \emph{Riemannin summa}
    \[
    R(f, \set{R_{i j}}, \set{\xi_{i j}}) = \sum_{i,j} f(\xi_{i j}) \Delta(R_{i j}).
    \]
\end{definition}

\begin{definition}[Riemann-integroituva]
    \label{integroituva}
    \uses{riemannin_summa, jaon_normi}
    Rajoitettu funktio $f$ on \emph{Riemann-integroituva} suorakaiteen $D$ yli, jos on olemassa $I \in \R$ siten että kaikilla $\epsilon \gt 0$ on olemassa $\delta \gt 0$ siten että kaikilla $m, n \in \N_1$ ja $\set{R_{i j}}_{1 \le i \le m, 1 \le j \le n}$ joille $\norm{\set{R_{i j}}} \lt \delta$, niin kaikilla $\set{\xi_{i j}}_{1 \le i \le m, 1 \le j \le n}$ joille $\xi_{i j} \in R_{i j}$ kaikilla $i, j$, pätee $|R(f, \set{R_{i, j}}, \set{\xi_{i, j}})| \lt \epsilon$.
\end{definition}

\begin{remark}
    Kurssin määritelmässä 4.1.1 Riemann-integroituvuudesta on $\xi$ jätetty tahallaan kvantifioimatta, joka aiheuttaa vaikeuden tulkita määritelmä oikein.

    Määritelmä \ref{integroituva} vastaa kurssin määritelmää, kunhan tulkitaan, että $\set{\xi_{i j}}$ kvantifioidaan (universaalisti) vasta $\set{R_{i j}}$ kvantifioinnin jälkeen.
    Tämä tulkinta ei ole yksiselitteinen, koska Riemannin summan määritelmä ei eksplisiittisesti vaadi $\set{\xi_{i j}}$ riippuvan osituksesta $\set{R_{i j}}$.
    Lisäksi määritelmästä 4.1.1 ei ole selvää kvantifioidaanko $\set{\xi_{i j}}$ universaalisti vai eksistentiaalisti.

    Normaalisti matematiikassa implisiittiset parametrit sidotaan universaalisti lauseen tai määritelmän ylätasolla, mutta tämä johtaa väärään tulkintaan määritelmän 4.1.1 kohdalla.
\end{remark}

\section{Integrointi yleisten joukkojen yli}

\begin{definition}[Nollajoukko]
    Joukko $N \subseteq \R^2$ on \emph{nollajoukko} jos jokaiselle $\epsilon \gt 0$ on olemassa (mahdollisesti äärellinen) jono suorakaiteita $R_i = [a_i, b_i] \times [c_i, d_i]$ jotka peittävät $N$ siten että

    \[
    \sum_{i} \area(R_i) \lt \epsilon.
    \]
\end{definition}

\begin{definition}[Kompakti]
    Joukko $A \subseteq \R^n$ on \emph{kompakti}, jos jokaisella jonolla $(x_i)$, missä $x_i \in A$ kaikilla $i$, on sellainen suppeneva osajono $(x_{i_j})$ että $\lim_{j \to \infty} x_{i_j} \in A$.
\end{definition}

\begin{definition}[Suljettu]
    Joukko $A \subseteq \R^n$ on \emph{suljettu}, jos jokaisella suppenevalla jonolla $(x_i)$, missä $x_i \in A$ kaikilla $i$, pätee $\lim_{i \to \infty} x_i \in A$.
\end{definition}

\begin{theorem}[Heine-Borel]
    Joukko $A \subseteq \R^n$ on kompakti jos ja vain jos se on rajoitettu ja suljettu.
\end{theorem}

\begin{example}[Pohdintatehtävä]
    Olkoot $g_1, g_2 : [a, b] \to \R$ jatkuvia ja $g_1 \leq g_2$.
    Osoita, että

    \[
    A = \set{(x, y) \mid g_1(x) \leq y \le g_2(x)}
    \]

    on kompakti.
\end{example}
\begin{proof}
    Osoitetaan, että $A$ on suljettu.
    Olkoon $(s_i)$ suppeneva jono $A$ pisteitä.

    \[
        \vdash \lim_{i \to \infty} s_i \in A.
    \]

    Nimetään suppeneva piste $(x, y)$

    \[
        \vdash \lim_{i \to \infty} s_i = (x, y) \
    \]

    Riittää osoittaa komponenttifunktioiden suppeneminen

    
    \[
        s_i = (x_i, y_i)
    \]
    \[
        \vdash \lim_{i \to \infty} x_i = x \land
        \lim_{i \to \infty} y_i = y
    \]
    
    Huomataan, että $\set{x_1 \mid x \in A} = [a, b]$, sillä kaikilla $x$ on olemassa $y$ siten että $g_1(x) \leq y \leq g_2(x)$.
    Sillä $[a, b]$ on suljettu ja $(x_i)$ suppenee, niin ensimmäinen konjunktion jäsen on totta.

    \[
    \vdash \lim_{i \to \infty} y_i = y
    \]

    Sillä $g_1$ jva, niin $g_1(x) = \lim_{i \to \infty} g_1(x_i)$.
    Vastaavasti $g_2(x) = \lim_{i \to \infty} g_2(x_i)$.
    Nyt, koska $\lim_{i \to \infty} y_i = y$

    \[
    g_1(x) = \lim_{i \to \infty} g_1(x_i) \leq \lim_{i \to \infty} y_i = y = \lim_{i \to \infty} y_i \leq \lim_{i \to \infty} g_2(x_i) = g_2(x)
    \]

    Täten $(x, y) \in A$.
\end{proof}

\begin{theorem}[Tyhjä joukko on nollajoukko]
    
\end{theorem}
\begin{proof}
    
\end{proof}

\begin{theorem}[Integraalin additiivisuus]
    
\end{theorem}

\section{Epäoleelliset integraalit}

\begin{remark}
    Palautetaan mieleen 1. ulottuvuuden tapaus.
    Jos $f : (0, 1] \to \R$ ja se on integroituva jokaisella välillä $[t, 1]$ kun $0 < t < 1$.

    \[
    \int_{0}^{1} f(x) dx = \lim_{t -> 0+} \int_{t}^{1} f(x) dx
    \]

    Kutsutaan tällaista epäoleelliseksi integraaliksi
\end{remark}

\section{Muuttujan vaihto integraaleissa}